\usepackage[svgnames]{xcolor}		% For color names

\usepackage[vietnamese]{babel}		% For the content
%\usepackage{vntex}					% For the title, captions

\usepackage{times}
\usepackage[unicode]{hyperref}		% For hyperlinks

% --------------------------------------------------------------------------------------
% Packages for Mathematics
\usepackage{mtpro2}
% ----- Neu khong dung goi mtpro2 ----- %
%\usepackage{mathptmx}
%\usepackage{amssymb}
% --------------------------------------------------------------------------------------

\usepackage[centertags]{amsmath}
\usepackage{amsfonts} 				% For maths fonts
\usepackage{amsthm} 				% For maths theorem

\usepackage{tikz}					% For plotting charts/figures/diagrams/etc.
\usetikzlibrary{shapes,snakes}
\usetikzlibrary{positioning}
\usetikzlibrary{arrows}

% --------------------------------------------------------------------------------------
% Packages for references, biographies
% --------------------------------------------------------------------------------------
\usepackage[square,numbers,sectionbib]{natbib}	% For chapter refs
\usepackage{chapterbib}

% --------------------------------------------------------------------------------------
% Packages for Layout 
% --------------------------------------------------------------------------------------
\usepackage[width=16cm,height=24cm,top=3cm]{geometry}
\usepackage{caption}			% For captions
\usepackage{wrapfig}			% For figures
\usepackage{mdframed}			% For sidenotes and other frames
\usepackage{wasysym}			% For emoticons. Notes: This package must be use after the maths packages.
\usepackage{background}			% For background
\usepackage{wallpaper}			% For background images
\usepackage{fancyhdr}			% For margins, headers, footers
\usepackage{titletoc}			% For table of contents
\usepackage{titlesec}
\usepackage{suffix}				% For authors in TOC
\usepackage{tocloft}
\usepackage{xparse}
\usepackage{xspace}
\usepackage{pdfpages}
\usepackage{listings}

% Tăng khoảng cách của new line
\usepackage{parskip}

% Thụt đầu dòng cho enumerate
\usepackage[shortlabels]{enumitem}

\usepackage{array}
\usepackage{makecell}
\newcommand\diag[4]{%
  \multicolumn{1}{|p{#2}|}{\hskip-\tabcolsep
  $\vcenter{\begin{tikzpicture}[baseline=0,anchor=south west,inner sep=#1]
  \path[use as bounding box] (0,0) rectangle (#2+2\tabcolsep,\baselineskip);
  \node[minimum width={#2+2\tabcolsep-\pgflinewidth},
        minimum  height=\baselineskip+\extrarowheight-\pgflinewidth] (box) {};
  \draw[line cap=round] (box.north west) -- (box.south east);
  \node[anchor=south west] at (box.south west) {#3};
  \node[anchor=north east] at (box.north east) {#4};
 \end{tikzpicture}}$\hskip-\tabcolsep}}
\newcolumntype{x}[1]{>{\centering\arraybackslash}p{#1}}
