% ---- frame problem ----
\mdfdefinestyle{frame}{
	linewidth=3pt,
	linecolor=gray,
	backgroundcolor=gray!6,
	topline=false,
	bottomline=false,
	rightline=false,
	innerleftmargin=0.5em
}
\surroundwithmdframed[style=frame]{}
% ---- end frame ----

% \newcommand{\code}[1]{\texttt{#1}}
\newcommand{\hh}{\usefont{T5}{iwona}{b}{n}}
\newcommand{\cmss}{\usefont{T5}{cmss}{bx}{n}}
\newcommand{\iwona}{\usefont{T5}{iwona}{l}{n}}

\newtheoremstyle{newstyle}
	{3pt} % Space above
	{5pt} % Space below
	{} % Body font
	{} % Indent amount
	{} % Theorem head font
	{.} % Punctuation after theorem head
	{.5em} % Space after theorem head
	{} % Theorem head spec (can be left empty, meaning `normal')

\theoremstyle{plain}
\newtheorem*{pro_no_count}{\cmss\problemColor Problem}

\theoremstyle{newstyle}
\newtheorem{bode}{\cmss Bổ đề}
\newtheorem{baitoan}{\cmss\problemColor Bài}
\newtheorem*{bt}{\cmss\problemColor Bài toán}

\theoremstyle{definition}
\newtheorem{pro}{\cmss\problemColor Problem}
\newtheorem{baitap}{\cmss\problemColor Bài tập}
\newtheorem*{nhanxet}{\cmss\problemColor Nhận xét}

\def\ge{\geqslant}
\def\le{\leqslant}
\def\geq{\geqslant}
\def\leq{\leqslant}

% Tăng khoảng cách giữa 2 {pro}
\makeatletter
\patchcmd{\endpro}{\endtrivlist}{\endtrivlist\vspace{0.6em}}{}{}
\makeatother

% ---- code ---- %
\usepackage[T1]{fontenc}
% \usepackage[formats]{listings}

\definecolor{backgroundcolor}{gray}{0.97}
\definecolor{indigo(dye)}{rgb}{0.0, 0.25, 0.42}

\def\lcode#1{{\ttfamily\detokenize{#1}}} % code in line
\def\code#1{{\ttfamily\detokenize{> #1}}}

\lstset {
  	tabsize			= 4,
	breaklines		= true,
	frame			= single,
	columns			= fullflexible,
    backgroundcolor	= \color{backgroundcolor},
    basicstyle		= \footnotesize\ttfamily  	
}

% Keywords here high line 
\lstset {
	emph=[1] {
       local, global, proc, for, from, to, do, end, if, else, fi, then, elif, return
    },
    emphstyle=[1]{\color{indigo(dye)}\bfseries},
    %
    emph=[2]% Variable Types
    {% 
        `:=`,
    },
  emphstyle=[2]{\color{blue}},
}
% ---- end code ---- 

% Dèine -> LastUpdated
\newcommand{\lastupdated}{%
	\begin{flushright}
		\def\parsedate ##1:20##2##3##4##5##6##7##8\empty{20##2##3/##4##5/##6##7}%
		\def\moddate##1{\expandafter\parsedate\pdffilemoddate{##1}\empty}%
		\cmss Last Updated: \moddate{\jobname.tex}%
	\end{flushright}
}
